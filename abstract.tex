\begin{abstract}

Recent new hardware and technology enable low-power inexpensive distributed
sensor networks. To realize certain applications such as real time event
detection, target tracking and system monitoring, time synchronization is
essential. In the literature, a book of time synchronization techniques have
been proposed. In general, the goal of a time synchronization mechanism is to
devise a scheme that improves accuracy with lower energy consumption. Yet,
there is clearly a trade off between accuracy and energy consumption. It would
not be always desirable to deploy the most accurate time synchronization as it
might drain the batteries or clog the network with time synchronization
messages. The choice of a time synchronization mechanism will depend on
application's demand on timing accuracy as well as the energy budget. To
estimate application's demand and select a proper time synchronization
mechanism, the relationship between application's performance or QoI (Quality
of Information) and time synchronization services are yet to be investigated.
This paper formalizes the relationship between the performance of applications
and time synchronization services based on analytic framework using
representative applications, namely, event detection and estimation. The
analysis shows the impact of timing errors for different event duration, target
moving speed, number of sensors, sampling frequencies, etc. The analysis
framework can also be used to extract the maximum synchronization errors that
each application can sustain to achieve the desired QoI.

Moreover, time synchronization plays a key role in energy-saving requirements.
The intuitive assumption that using more temperature stable clocks will
automatically improve duty cycling performance, and thus decrease power
consumption, does not always hold true. In this article, we will present the
link between clock stability, impact on duty cycling, and the possible bandwidth
savings that can be made by using temperature compensated clocks or clock drift
estimation techniques.
\end{abstract}
