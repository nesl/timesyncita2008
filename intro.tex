\section{Introduction}
Recent new hardware and technology enable low-power inexpensive
distributed sensor networks. To realize certain applications such as
real time event detection, target tracking and system monitoring,
time synchronization is essential. In the literature, a book of time
synchronization techniques have been proposed including Time-Sync
Protocol for Sensor Network (TPSN)~\cite{ganeriwal03timingsync},
Reference Broadcast Synchronization (RBS)
~\cite{elson02finegrained}, elapsed time \cite{kusy05elapsed}, etc.
In general, the goal of a time synchronization mechanism is to
devise a scheme that improves accuracy with lower energy
consumption. Yet, there is clearly a trade off between accuracy and
energy consumption. It would not be always desirable to deploy the
most accurate time synchronization as it might drain the power. The
choice of a time synchronization mechanism will depend on
application's demand on timing accuracy as well as the energy
budget. To estimate application's demand and select a proper time
synchronization mechanism, the relationship between application's
performance or QoI (Quality of Information) and time synchronization
services are yet to be investigated. In this paper, we aim to
formalize the relationship between the performance of applications
and time synchronization services based on analytic frameworks using
representative applications, namely, event detection and estimation.

%what's the difference/contribution of our paper from existing survey papers on time sync?

Moreover, time synchronization may play a key role in energy-saving
requirement. For example, accurate time synchronization can improve
the efficiency of duty-cycling and thus save energy.
Even though our study implies that energy saving by improving duty
cycling through better clock accuracy would not be significant with
low-power sensor platforms (e.g., motes?). We will show the analysis
with regard to energy saving in Section \ref{sec:power}.

%The rest of the paper is organized as the follows. In Section \ref{},
%we will first look at the application QoI impact related to the
%timing errors. In section 3, we will present the relationship
%between timing accuracy and duty cycling in terms of energy saving.
