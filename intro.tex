\section{Introduction}
Recent new hardware and technology enable low-power inexpensive
distributed sensor networks. To realize certain applications such as
real time event detection, target tracking and system monitoring,
time synchronization is essential. In the literature, a book of time
synchronization techniques have been proposed including Time-Sync
Protocol for Sensor Network (TPSN)~\cite{ganeriwal03timingsync},
Reference Broadcast Synchronization (RBS)
~\cite{elson02finegrained}, elapsed time \cite{kusy05elapsed}, etc.
In general, the goal of a time synchronization mechanism is to
devise a scheme that improves accuracy with lower energy
consumption. However, there is clearly a trade off between accuracy and
energy consumption. It would not always be desirable to deploy the
most accurate time synchronization as it might drain the batteries too fast. The
choice of a time synchronization mechanism will depend on the
application's demand on timing accuracy as well as the available energy
budget. To estimate an application's demand and select a proper time
synchronization mechanism, the relationship between application
performance or QoI (Quality of Information) and time synchronization
services are yet to be investigated. In this paper, we aim to
formalize the relationship between the performance of applications
and time synchronization services based on an analytic framework using
representative applications, namely, event detection and estimation.

%what's the difference/contribution of our paper from existing survey papers on time sync?

Time synchronization may play a key role in energy-saving requirements. For
example, accurate time synchronization can improve the efficiency of
duty-cycling and thus save energy. However, different clock sources consume
different amounts of energy. In general, more temperature stable oscillators
consume more energy. Our analysis of clock stability and duty cycling
performance will show how much energy a duty-cycled system can gain by employing
a higher stability clock, and vice-versa it shows the maximum energy such a
clock system is allowed to use, in order not to offset the gained energy.
Additionally, the local clock stability is tightly bound to the number of times
a time synchronization service has to resynchronize with its peers. We will
investigate upper and lower bounds on the number of resynchronization requests
using a 3 year temperature data set and modeling crystal drifts.


